\section*{Problemas propuestos}
\begin{pbm}
	Muestre que la parte fraccionaria del n\'umero $\sqrt{4n^{2} + n}$ no es mayor a $0,25$
\end{pbm}
\begin{pbm}
	Sean $\{a_{i}\}_{1 \leq i \leq r}$ enteros no negtivos con $n = a_{1} + a_{2} + \cdots + a_{r}$. Muestre que $\dfrac{n!}{a_{1}!a_{2}!\cdots a_{r}!}$ es un entero.
\end{pbm}
\begin{pbm}
	Pruebe que, para cualquier $n \in \N$, $\displaystyle\sum_{i=1}^{\infty}\biggl\lfloor \dfrac{n + 2^{i-1}}{2^{i}} \biggr\rfloor = n$
\end{pbm}
\begin{pbm}
	Sea $p$ un divisor primo del n\'umero $\dbinom{2n}{n}$, con $p \geq \sqrt{2n}$. Entonces el exponente de $p$ en la factoraci\'on en primos del n\'umero $\dbinom{2n}{n}$ es igual a $1$. 
\end{pbm}
\begin{pbm}[Corea 1997]
	Exprese $\displaystyle\sum_{k = 1}^{n}\lfloor \sqrt{k} \rfloor$ en t\'erminos de $n$ y $\lfloor \sqrt{n} \rfloor$.
\end{pbm}
\begin{pbm}[Canad\'a 1998]
	Determine el n\'umero de soluciones reales de la ecuaci\'on 
	$$\Bigl\lfloor \frac{a}{2} \Bigr\rfloor + \Bigl\lfloor \frac{a}{3} \Bigr\rfloor + \Bigl\lfloor \frac{a}{5} \Bigr\rfloor = a.$$
\end{pbm}
\begin{pbm}
	Encuentre todos los reales $\alpha$ tales que la igualdad\\
	$\lfloor \sqrt{n} \rfloor + \lfloor \sqrt{n + \alpha} \rfloor = \lfloor \sqrt{4n + 1} \rfloor$ es verdadera para todos los naturales $n$.
\end{pbm}
\begin{pbm}
	Si $a,b,c$ son reales y $\lfloor na \rfloor + \lfloor nb \rfloor = \lfloor nc \rfloor$ para todo $n \in \N$, entonces $a \in \Z$ o $b \in \Z$.
\end{pbm}
\begin{pbm}
	Sean $a,b,c$ y $d$ n\'umeros reales. Suponga que $\lfloor na \rfloor + \lfloor nb \rfloor = \lfloor nc \rfloor + \lfloor nd \rfloor$ para todo $n \in \Z^{+}$. Muestre que por lo menos uno de entre $a + b$, $a - c$, $a - d$ es entero. 
\end{pbm}
\begin{pbm}
	Sea $n \geq 3$ un entero. Muestre que es posible eliminar como m\'aximo dos elementos del conjunto $\{1,2,\cdots,n\}$ de modo que la suma de los n\'umeros restantes sea un cuadrado perfecto.
\end{pbm}
\begin{pbm}
	Sean $a,b,m$ enteros dados, con $mcd(a,m) = 1$. C\'alcule $\displaystyle\sum_{x = 0}^{m - 1}\biggl\lfloor \frac{ax + b}{m} \biggr\rfloor$.
\end{pbm}
\begin{pbm}
	Encuentre todos los naturales $n$ tales que $2^{n - 1}\mid n!$.
\end{pbm}
\begin{pbm}
	Determine los pares $(a,b)$ de reales tales que $a\lfloor bn \rfloor = b\lfloor an \rfloor$ para todo $n \in \Z^{+}$.
\end{pbm}
\begin{pbm}
	Si $p$ es primo, entonces $\dbinom{p^{k}}{i} \equiv 0 \Mod p$ para\\
	$1 \leq i \leq p^{k} - 1$.
\end{pbm}
\begin{pbm}
	Pruebe que $\lfloor(\sqrt[3]{n} + \sqrt[3]{n + 2})^{3}\rfloor$ es divisible por $8$.
\end{pbm}
\begin{pbm}
	Pruebe que $t_{1} + t_{2} + \cdots + t_{n} = \biggl\lfloor \dfrac{n}{1} \biggr\rfloor + \biggl\lfloor \dfrac{n}{2} \biggr\rfloor + \cdots + \biggl\lfloor \dfrac{n}{n} \biggr\rfloor$, donde $t_{n}$ es el n\'umero de divisores del natural $n$.
\end{pbm}
\begin{pbm}
	Pruebe que si $p$ es primo, entonces la diferencia $\dbinom{n}{p} - \biggl\lfloor \dfrac{n}{p} \biggr\rfloor$ es divisible por p.
\end{pbm}